\subsection{A Level Standard}

\begin{longtable}{ | p{2cm} | p{10cm} | p{3cm} | } 
    \hline
     & Technical Skill & Evidence\\
    \hline 
    Group A & & \\
    \hline
    1.1 & Complex User Defined Algorithm: a complex, heavily recursive Pratt parsing algorithm will be used to convert a linked list of Tokens into an Abstract Syntax Tree data structure representing the program and order of operations within. & \\
    \hline
    1.2 & A tree data structure will be used to store parsed nodes in an Abstract Syntax Tree. & \\
    \hline
    1.3 & Graph traversal algorithms will be required when optimising and compiling programs in order to locate any unconnected branches (redundant code optimisation) or to evaluate the result of an arithmetic operation at compile time, substituting it with a constant. & \\
    \hline
    1.4 & A stack data structure will be used to store return addresses of the PC (program counter) from subroutines in the emulator during \texttt{call} and \texttt{ret} instructions. & \\
    \hline
    1.5 & Complex user-defined use of object oriented programming using both composition, aggregation, and interfaces to relate common behaviour between distinct error structs. & \\
    \hline
    1.6 & Dynamic Generation of objects will be required in both the parser and the lexer when compiling user's programs and representing them as nodes or tokens. & \\
    \hline
    Group B & & \\
    \hline
    2.1 & Source code programs are read in from a text file and compiled, the machine code programs are written to a binary file & \\
    \hline
\end{longtable}
